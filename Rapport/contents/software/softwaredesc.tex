\section{Description of the software structure and functionality}

TBD Softwarebeskrivelse og underafsnit

\subsection {Description of the PID controller} 
 
A PID controller continuously calculates an error value as the difference to a reference point and a measured process variable.\\
PID is an abbreviation for a proportional-integral-derivative controller, it is a control loop feedback mechanism. The controllers job is to minimize the error value for the given devices running time. In the case of this project the reference point is the line and the PID will power up the engines to steer accordingly to said reference point.
$$\mathrm{F}(t)=K_p{e(t)} + K_{i}\int_{0}^{t}{e(\tau)}\,{d\tau} + K_{d}\frac{de(t)}{dt}$$
\subsubsection {Proportional control(P)}

The proportional term produces an output value that is proportionally related to the current error value, this value can be adjusted by modifying the error by a constant (Kp). A high proportional gain results in a large change in the output for a given change in the error. 

$$ P_{\mathrm{out}}=K_p\,{e(t)}$$  

If the proportional gain is too high, the system can become unstable. Contrarily, a small gain will result in the device adjusting too slowly, which decreases overall efficiency and in the case of this project, it will end up being detrimental to the steering accuracy.
\subsubsection {Integral control(I)}

The integral controller is contributing proportionally to both the magnitude of the error and the duration of the error. \\
The integral in a PID controller is the sum of the instantaneous error over time and gives the accumulated offset that should have been corrected previously. \\
The controller output equals the accumulated error multiplied by the integral gain(Ki)
$$I_{\mathrm{out}}=K_{i}\int_{0}^{t}{e(\tau)}\,{d\tau} 	$$ 

The integral term accelerates the movement of the process towards the reference point.
However, since the integral term responds to accumulated errors from the past, it can cause the present value to overshoot the reference value.
\subsubsection {Derivative control(D)} 
The derivative of the process error is calculated by determining the slope of the error over time and multiplying this rate of change by the derivative gain Kd. The magnitude of the contribution of the derivative term to the overall control action is termed the derivative gain, Kd.

The derivative term is given by:

$$D_{\mathrm{out}}=K_d\frac{de(t)}{dt}$$
The derivative action predicts system behaviour and utilizes this to improve the settling time and stability of the system.
An ideal derivative is not causal, so that implementations of PID controllers include an additional low pass filtering for the derivative term, to limit the high frequency gain and noise.


%Derivative action is seldom used in practice though - by one estimate in only 25$\%$ of deployed controllers  because of its variable impact on system stability in real world applications.