\section{Description of the hardware structure and functionality}
\subsection{Selection of sensor}
\begin{table}[htbp]
    \begin{tabular}{|l|l|l|l|}
        \hline
        Name                & QRE1113 board & OPB706A  & OPB704    \\ \hline
        Max sensor distance & 3mm                            & 1.27mm   & 3.8mm     \\ \hline
        Forward current     & 50mA                           & 20mA     & 40mA      \\ \hline
        Mounting            & On print                       & THT      & In casing \\ \hline
        Price               & 19.43DKK                       & 26.90DKK & 42.55DKK  \\ \hline
        Notes               & ~                              & ~        & ~         \\
        \hline
    \end{tabular}
    \caption{Table of a selection of sensors}
\label{sensor_tabel}
\end{table}
TBD Beskriv sensorer og hvorfor vi har valgt denne
\newline
\subsection{OPB704 Sensor}
Sensor choice for the line following robot will be the OPB704. The choice was logical because of the availability. It was confirmed in the brainstorming process that the sensor was reliable and had a wide range of applications. 3D printing the mount was ideal for the purpose of the sensor and the task ahead.
\newline 
\\
TBD Beskriv hardware 
struktur og funktion samt alle underdele


\subsection{Analog-to-digital converter (ADC)}

The purpose of the ADC is to convert the analog data from the sensors to digital data that can be manipulated by a computer - this allows data received from the bluetooth transmitter on the robot to be processed into readable data more easily, so we can understand how the sensors are reacting to the environment. The sensors themselves cannot discern what they actually need to read, they just read anything they can see and send that signal. \\
Analog signals can have a significant amount of noise - since any received noise is interpreted as part of the signal, so a digital signal is not only more easy to work with, it will also have less chaotic frequencies. This will make for more accurate readings on the tachometer on the robot, which allows even more finely tuned monitoring of the robot and its working processes. \\

TBD ADC diagram\\

TBD Our take on the ADC

