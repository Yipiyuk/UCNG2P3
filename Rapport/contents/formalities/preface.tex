\chapter*{Introduction}


%Læsevejledning:\\
%Kommer senere i projektforløbet.
%\\\\
A line following robot is basically a robot designed for the consumer to follow a line or path that is predetermined. This line or path may be as simple as a strip of tape or a black line and if  developed further can follow e.g magnetic markers, embedded lines and laser guided markers. In order to detect the various lines or paths, miscellaneous sensors or sensing schemes can be used.\newline These schemes may range from simple low cost schemes to advanced and more expensive vision systems. In the industry the many different types of robots are already implemented in semi to fully automatic systems.\\

The project was handed to the group april 12'th and will be handed in at UCN Sofiendalsvej, june 7'th at 12.00.\newline
The objective of this project is to design and implement an automotive robot capable of autonomous maneuvering, specifically a line-following robot employing light detecting sensors.  \\
The challenges at hand is to design a system for the board, to utilize the ADC capabilities of the chip and to implement a PID controller. Furthermore test the products performance on a test track to optimize by trial and error and to implement software with multiple light detecting sensors.


%
\phantom{Luft}\vspace{3cm}
\begin{table}[H]
	\centering
		\begin{tabular}{c c c}
			\underline{\phantom{JAERJAERJAERJAERGO}} & \phantom{cookies} & \underline{\phantom{JAERJAERJAERJAERGO}} \\
			Benjamin Nielsen			& \phantom{cookies} & Henrik Jensen		\\
			&&\\
			&&\\
			\underline{\phantom{JAERJAERJAERJAERGO}} & \phantom{cookies} & \underline{\phantom{JAERJAERJAERJAERGO}} \\
			Martin Nonboe			& \phantom{cookies} & Nikolaj Bilgrau		\\
			&&\\
						
		\end{tabular}
\end{table}