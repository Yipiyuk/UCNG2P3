\newpage
\section{C\# code - interface}
\begin{lstlisting}
using System;
using System.Collections.Generic;
using System.ComponentModel;
using System.Data;
using System.Drawing;
using System.Linq;
using System.Text;
using System.Threading.Tasks;
using System.Windows.Forms;
using System.IO.Ports;

namespace Line_follower_Telemetry
{
    public partial class Main : Form
    {
        int[] baudRates = { 2400, 4800, 9600, 19200, 38400, 57600, 115200 };
        string[] ports = SerialPort.GetPortNames();
        
        public delegate void SCDelegate();
        SerialPort SP = new SerialPort();
        delegate void printDelegate(string data);

        int VDA = 0, VDB = 0, HDA = 0, HDB = 0;

        int res = 0;
        int resMin = -3;
        int resMax = 3;
        int resPosOffset = 0;

        public Main()
        {
            InitializeComponent();
            for (int i = 0; i < baudRates.Length; i++)
            {
                BaudCB.Items.Add(baudRates[i]);
            }
            BaudCB.SelectedItem = 115200;
            PortsCB.Items.AddRange(ports);
            PortsCB.SelectedIndex = 0;
            Point resP = ResPositive.PointToScreen(Point.Empty);
            resPosOffset = resP.X-8;
            resLabel.Text = resPosOffset.ToString();
        }

        private void Main_Load(object sender, EventArgs e)
        {

        }

        void testPrint(string data)
        {
            RawSerial.Text = data;
            serialDeconstruct(data);
            if(data[data.Length - 1] == 'Z')
            {
                RawSerial.AppendText("\n");
            }
            RawSerial.SelectionStart = RawSerial.Text.Length;
            RawSerial.ScrollToCaret();
            
        }

        private void serialDeconstruct(string data)
        {
            string[] dataP = data.Split(new string[] { "|" }, StringSplitOptions.None);
            if (dataP.Length == 17)
            {
                vPWMLabel.Text = dataP[4];
                if(Convert.ToInt32(dataP[4]) > 0 && Convert.ToInt32(dataP[4]) < 150)
                {
                    vPWM.Height = map(Convert.ToInt32(dataP[4]), 0, 100, 150, 0);
                    backpanel1.BackColor = Color.FromArgb(map(vPWM.Height, 150, 0, 0, 255), map(vPWM.Height, 150, 0, 255, 0), 0);
                }
                VDA = Convert.ToInt32(dataP[2]);
                if (VDA == 1) vDirA.BackColor = Color.Green;
                else vDirA.BackColor = Color.White;
                VDB = Convert.ToInt32(dataP[3]);
                if (VDB == 1) vDirB.BackColor = Color.Green;
                else vDirB.BackColor = Color.White;
                hPWMLabel.Text = dataP[8];
                if(Convert.ToInt32(dataP[8]) > 0 && Convert.ToInt32(dataP[8]) < 150)
                {
                    hPWM.Height = map(Convert.ToInt32(dataP[8]), 0, 100, 150, 0);
                    Backpanel2.BackColor = Color.FromArgb(map(hPWM.Height, 150, 0, 0, 255), map(hPWM.Height, 150, 0, 255, 0), 0);
                }
                HDA = Convert.ToInt32(dataP[6]);
                if (HDA == 1) hDirA.BackColor = Color.Green;
                else hDirA.BackColor = Color.White;
                HDB = Convert.ToInt32(dataP[7]);
                if (HDB == 1) hDirB.BackColor = Color.Green;
                else hDirB.BackColor = Color.White;
                Sensor1Label.Text = dataP[9];
                Sensor1.Height = map(Convert.ToInt32(dataP[9]), 0, 1023, 100, 0);
                Backpanel3.BackColor = Color.FromArgb(map(Sensor1.Height, 100, 0, 0, 255), map(Sensor1.Height, 100, 0, 255, 0), 0);
                Sensor2Label.Text = dataP[10];
                Sensor2.Height = map(Convert.ToInt32(dataP[10]), 0, 1023, 100, 0);
                Backpanel4.BackColor = Color.FromArgb(map(Sensor2.Height, 100, 0, 0, 255), map(Sensor2.Height, 100, 0, 255, 0), 0);
                Sensor3Label.Text = dataP[11];
                Sensor3.Height = map(Convert.ToInt32(dataP[11]), 0, 1023, 100, 0);
                Backpanel5.BackColor = Color.FromArgb(map(Sensor3.Height, 100, 0, 0, 255), map(Sensor3.Height, 100, 0, 255, 0), 0);
                Sensor4Label.Text = dataP[12];
                Sensor4.Height = map(Convert.ToInt32(dataP[12]), 0, 1023, 100, 0);
                Backpanel6.BackColor = Color.FromArgb(map(Sensor4.Height, 100, 0, 0, 255), map(Sensor4.Height, 100, 0, 255, 0), 0);
                Sensor5Label.Text = dataP[13];
                Sensor5.Height = map(Convert.ToInt32(dataP[13]), 0, 1023, 100, 0);
                Backpanel7.BackColor = Color.FromArgb(map(Sensor5.Height, 100, 0, 0, 255), map(Sensor5.Height, 100, 0, 255, 0), 0);
                Sensor6Label.Text = dataP[14];
                Sensor6.Height = map(Convert.ToInt32(dataP[14]), 0, 1023, 100, 0);
                Backpanel8.BackColor = Color.FromArgb(map(Sensor6.Height, 100, 0, 0, 255), map(Sensor6.Height, 100, 0, 255, 0), 0);
                Sensor7Label.Text = dataP[15];
                Sensor7.Height = map(Convert.ToInt32(dataP[15]), 0, 1023, 100, 0);
                Backpanel9.BackColor = Color.FromArgb(map(Sensor7.Height, 100, 0, 0, 255), map(Sensor7.Height, 100, 0, 255, 0), 0);
                res = Convert.ToInt32(dataP[5]);
                resLabel.Text = res.ToString();
                if (res < 0 && res >= resMin)
                {
                    ResBack.BackColor = Color.FromArgb(map(res, resMin, 0, 255, 0), map(res, resMin, 0, 0, 255), 0);
                    ResNegative.Width = map(res, 0, resMin, 248, 0);
                } else if (res > 0 && res <= resMax)
                {
                    ResBack.BackColor = Color.FromArgb(map(res, resMax, 0, 255, 0), map(res, resMax, 0, 0, 255), 0);
                    ResPositive.Width = map(res, 0, resMax, 248, 0);
                    ResPositive.Left = resPosOffset + ((248 / 3) * res);
                } else if (res == 0)
                {
                    ResPositive.Width = 248;
                    ResNegative.Width = 248;
                    ResPositive.Left = resPosOffset;
                }
            }
        }

        private void SPPrint(object sender, SerialDataReceivedEventArgs e)
        {

            SerialPort SP = (SerialPort)sender;
            printDelegate PD = new printDelegate(testPrint);
            this.Invoke(PD, SP.ReadLine());
        }

        private void SPSPrint(object sender, RTSerialCom.DataStreamEventArgs e)
        {
            RTSerialCom.SerialClient SPS = (RTSerialCom.SerialClient)sender;
            printDelegate PD = new printDelegate(testPrint);
            byte[] ByteArray = new byte[18];
            SPS.Receive(ByteArray, 0, ByteArray.Length);
            string data = System.Text.Encoding.Default.GetString(ByteArray);
            this.Invoke(PD, data);
        }

        public void SerialClose()
        {
            SP.DiscardInBuffer();
            SP.DiscardOutBuffer();
            SP.Close();
            SP.DataReceived -= new SerialDataReceivedEventHandler(SPPrint);
            ConBot.Text = "Connect";
        }

        private int map(int x, int in_min, int in_max, int out_min, int out_max)
        {
            return (x - in_min) * (out_max - out_min) / (in_max - in_min) + out_min;
        }

        private void ConBot_Click_1(object sender, EventArgs e)
        {
            if (SP.IsOpen)
            {
                BeginInvoke(new SCDelegate(SerialClose));
            }
            else
            {
                if (BaudCB.SelectedItem == null || PortsCB.SelectedItem == null)
                {
                    RawSerial.Text = "Parameters not set!";
                }
                else
                {
                    SP.BaudRate = (int)(BaudCB.SelectedItem);
                    SP.PortName = PortsCB.SelectedItem.ToString();
                    SP.Open();
                    SP.DataReceived += new SerialDataReceivedEventHandler(SPPrint);
                    ConBot.Text = "Disconnect";
                    RawSerial.Clear();
                }
            }
        }
    }
\end{lstlisting}