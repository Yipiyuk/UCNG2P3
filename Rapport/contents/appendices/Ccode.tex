\section{C code}
main.c:

\begin{lstlisting}
#include <xc.h>
#include <stdio.h>
#include <stdlib.h>
#include <stdint.h>
#define _SUPPRESS_PLIB_WARNING 1   
#define _DISABLE_OPENADC10_CONFIGPORT_WARNING 1  
#include <plib.h>
#include "setup.h"
#include "Functions.h"
#include "UART.h"
#include <math.h>
#include "Delay.h"
#include "ADC.h"

//Left wheel
#define DIRASetup TRISFbits.TRISF1
#define DIRA PORTFbits.RF1
//Right wheel
#define DIRBSetup TRISDbits.TRISD10
#define DIRB PORTDbits.RD10
//Bottom button
#define CalLSetup TRISDbits.TRISD8
#define CalL PORTDbits.RD8
//Top button
#define CalHSetup TRISDbits.TRISD9
#define CalH PORTDbits.RD9

int ADCHighestValue[7] = {0,0,0,0,0,0,0};
int ADCLowestValue[7] = {1024,1024,1024,1024,1024,1024,1024};
int caliFlag = 0;

//PWM
float kp = 8.5; 
float ki = 0.01;
float kd = 850; 
float lastError = 0;
float integral = 0;
int maxPwm = 80;
int initialPwm = 50;
int dir = 2; //0 = left, 1 = right
int turn = 0;
int DIR[] = {0, 0};

//PID variables
float sensorMean = 0;
float sensorSum = 0;
float sensorPos = 0;
float sensorProp = 0;
float sensorInt = 0;
float sensorDer = 0;
float sensorLastProp = 0;
float sensorError = 0;
float sensorOn[7] = {0,0,0,0,0,0,0};
int adjustedPWM[2] = {0,0};
int correctFlag = 0;

int main(int argc, char** argv) {
    int ADCs[] = {2, 8, 14, 3, 5, 9, 15}; //Pins for each sensor, in order
    int ADCAvgValue[7];

    int ADCData[7];
    int PWM[] = {0, 0};

    int POS[] = {0, 0};
    int i = 0;
    int averageFlag = 0;
    
    UART1Init(115200, 1);
    initPWM();
    DIRASetup = 0;
    DIRBSetup = 0;
    CalLSetup = 1;
    CalHSetup = 1;
    
    for (;;) {
        for(i = 0;i<7;i++){
            ADCData[i] = analogRead(ADCs[i]); //Reads data on all sensors
        }
        checkCalState(ADCs);
        //Working PID
        if(caliFlag == 1) //Run if calibrated
        {
            if(averageFlag == 0) //If average value on each sensor has not been calculated yet
            {
                for(i = 0;i<7;i++)
                    ADCAvgValue[i] = (ADCHighestValue[i]+ADCLowestValue[i])/2; //Calculate average value for each sensor
                averageFlag = 1;
            }
            for(i=0;i<7;i++){
                if(ADCData[i]>ADCAvgValue[i]) //If sensor detects a line, set this sensor to be 1 in sensorOn
                    sensorOn[i] = 1;
                else
                    sensorOn[i] = 0;
            }
            sensorMean = 0;
            sensorSum = 0;
            for(i=0;i<7;i++){
                sensorMean += sensorOn[i]*i*1; //Makes sensors ready for weighted mean calculation
                sensorSum += sensorOn[i]; //Calculate how many sensors are on
            }
            PID(sensorMean, sensorSum);//Calculates PID
            PWM[0] = adjustedPWM[0]; 
            PWM[1] = adjustedPWM[1];
            adjustDuty(1, PWM[0]); //Adjusts the duty cycle for left motor
            adjustDuty(2, PWM[1]); //Right motor
        }
        DIRA = DIR[0];
        DIRB = DIR[1];
        
        //Telemetry to interface
        telemetryOut(ADCData, PWM, DIR, POS); //Sends all data through UART
        DelayMs(12);
    }
    return (EXIT_SUCCESS);
}
\end{lstlisting}
\newpage
Functions.c:
\begin{lstlisting}
#include <xc.h>
#include "Delay.h"
#include <stdint.h>

#define CalL PORTDbits.RD8
#define CalH PORTDbits.RD9

extern int ADCHighestValue[];
extern int ADCLowestValue[];
extern int caliFlag;

extern float kp;
extern float ki;
extern float kd;
extern int initialPwm;
extern int maxPwm;
extern int dir;

extern float sensorPos;
extern float sensorProp;
extern float sensorInt;
extern float sensorDer;
extern float sensorLastProp;
extern float sensorError;
extern int adjustedPWM[2];

int calLFlag = 0;
int calHFlag = 0;

void dec_to_str(char* str, int val, size_t digits) {
    size_t i = 1u;
    for (; i <= digits; i++) {
        str[digits - i] = (char) ((val % 10u) + '0');
        val /= 10u;
    }
    str[i - 1u] = '\0'; // assuming you want null terminated strings?
}

void initPWM() {
    int sysClk = 80000000;
    int pwmFreq = 1000;
    int prescaleV = 1;
    int dutyCycle = 0;

    OC2CON = 0x0000;
    OC2R = 0x00638000;
    OC2RS = 0x00638000;
    OC2CON = 0x0006;
    T2CONSET = 0x0008;
    PR2 = (sysClk / (pwmFreq * 2) * prescaleV) - 1;
    OC2RS = (PR2 + 1)*((float) dutyCycle / 100);
    
    T2CONSET = 0x8000;
    OC2CONSET = 0x8020;

    OC4CON = 0x0000;
    OC4R = 0x00638000;
    OC4RS = 0x00638000;
    OC4CON = 0x0006;
    T4CONSET = 0x0008;
    PR4 = (sysClk / (pwmFreq * 2) * prescaleV) - 1;
    OC4RS = (PR4 + 1)*((float) dutyCycle / 100);
    
    T4CONSET = 0x8000;
    OC4CONSET = 0x8020;
    
}

void adjustDuty(int channel, int duty) {
    switch (channel) {
        case 1:
            OC2RS = (PR2 + 1)*((float) duty / 100);
            break;
        case 2:
            OC4RS = (PR4 + 1)*((float) duty / 100);
            break;
    }
}

void telemetryOut(int* ADCData, int* PWM, int* DIR, int* POS) {
    char str[4u + 1u];
    int i = 0;

    /* TRANSMISSION START */
    UART1Write("A|");
    
    dec_to_str(str, POS[0], 4u);
    UART1Write(str);
    UART1Write("|");
    if(DIR[0] == 0) UART1Write("0");
    else UART1Write("1");
    UART1Write("|");
    if(DIR[0] == 0) UART1Write("1");
    else UART1Write("0");
    UART1Write("|");
    dec_to_str(str, PWM[0], 3u);
    UART1Write(str);
    UART1Write("|");
    
    dec_to_str(str, POS[1], 4u);
    UART1Write(str);
    UART1Write("|");
    if(DIR[1] == 0) UART1Write("0");
    else UART1Write("1");
    UART1Write("|");
    if(DIR[1] == 0) UART1Write("1");
    else UART1Write("0");
    UART1Write("|");
    dec_to_str(str, PWM[1], 3u);
    UART1Write(str);
    UART1Write("|");
    
    for (i = 0; i < 6; i++) {
        dec_to_str(str, ADCData[i], 4u);
        UART1Write(str);
        UART1Write("|");
    }
    dec_to_str(str, ADCData[6], 4u);
    UART1Write(str);
    
    UART1WriteLn("|Z");
    /* TRANSMISSION END */
}

//Checks for calibration, if not, calibrate
void checkCalState(int* adc)
{
    int i;
    int j;
    int tempRead;
    if(CalL < 1){
       for(i = 0;i<10;i++){
           for(j = 0;j<7;j++){
               tempRead = analogRead(adc[j]);
               if(tempRead < ADCLowestValue[j])
                   ADCLowestValue[j] = tempRead;
           }
       }
       calLFlag = 1;
    }
    if(CalH < 1){
       for(i = 0;i<10;i++){
           for(j = 0;j<7;j++){
               tempRead = analogRead(adc[j]);
               if(tempRead > ADCHighestValue[j])
                   ADCHighestValue[j] = tempRead;
           }
       }
       calHFlag = 1;
    }
    if(caliFlag != 1){
        if(calLFlag == 1 && calHFlag == 1) {
            DelayMs(1500);
            caliFlag = 1;
        }
    }
}

void PID(int sensorMean, int sensorSum)
{
    int PWM[] = {0,0};
    if(sensorSum > 0) //As long as there is at least one sensor active
    {
        sensorPos = sensorMean/sensorSum; //Position of the line on the sensorarray
        sensorProp = sensorPos - 3;  //Proportional part, position minus middle sensor position
        sensorInt = sensorInt + sensorProp; //Integral part
        if(sensorInt > 100) //Makes sure Integral is not too large, reduces time to adjust
            sensorInt = 100;  
        if(sensorInt < -100)
            sensorInt = -100;
        sensorDer = sensorProp - sensorLastProp; //Derivative part
        sensorError = (sensorProp * kp)+(sensorInt*ki)+(sensorDer*kd); //PID calculation
        sensorLastProp = sensorProp; //Saves proportional for next derivative
        if(sensorError < -(initialPwm)) //Sets an upper cap for adjustment
            sensorError = -(initialPwm);
        if(sensorError > (initialPwm))
            sensorError = (initialPwm);
        if(sensorError<0){
            PWM[1] = initialPwm+sensorError; //Decrease right motor, sensorError is negative here
            PWM[0] = initialPwm-sensorError; //Increase left motor
            dir = 1; //turn right
        }
        else if(sensorError>0){
            PWM[1] = initialPwm+sensorError;
            PWM[0] = initialPwm-sensorError; 
            dir = 0; //turn left
        }
        else{
            PWM[0] = initialPwm;
            PWM[1] = initialPwm;
            dir = 2;
        }
    }
    else
    {
        PWM[0] = adjustedPWM[0];
        PWM[1] = adjustedPWM[1];
    }
    adjustedPWM[0] = PWM[0];
    adjustedPWM[1] = PWM[1];
}
\end{lstlisting}
\newpage
UART.c:
\begin{lstlisting}
#include <xc.h>

extern void UART1Init(int baudrate, int stopBit) {
    if (U1STAbits.OERR != 0) {              // Check recieve buffer for overflow
        U1STAbits.OERR = 0;                 // Reset flag if set
    }
    
    TRISBbits.TRISB4 = 0;                   // Set B4 as output
    U1MODEbits.BRGH = 1;                    // Baudrate generator high mode     
    U1MODEbits.PDSEL = 0;                   // Parity and data selection bits 8bit, no parity     
    if (stopBit == 1) {
        U1MODEbits.STSEL = 0;               // Stopbit set to 1 stopbit     
    } else {
        U1MODEbits.STSEL = 1;               // Stopbit set to 2 stopbit     
    }
    U1BRG = (40000000 / 4 / baudrate) - 1;  // Set baudrate register     
    U1STAbits.URXEN = 1;                    // Enable Reciever     
    U1STAbits.UTXEN = 1;                    // Enable Transmitter     
    U1MODEbits.ON = 1;                // Turn on USART 
}

extern char UART1Putc(char c) {
    while (U1STAbits.UTXBF);                // Waits for trasmit buffer to bet not full
    U1TXREG = c;                            // Puts contents of C into buffer
    return c;
}

extern void UART1WriteLn(char *data) {
    while (*data) UART1Putc(*data++);
    UART1Putc('\n');
}
\end{lstlisting}
\newpage
ADC.c:
\begin{lstlisting}
#include <xc.h>
#include "Delay.h"

extern int analogRead(char CH) {
    AD1PCFG = ~CH; // PORTB = Digital; RB2 = analog
    AD1CON1 = 0x0000; // SAMP bit = 0 ends sampling
    // and starts converting
    //AD1CHS = 0x00020000; // Connect RB2/AN2 as CH0 input
    // in this example RB2/AN2 is the input
    AD1CSSL = 0;
    AD1CON3 = 0x0001; // Manual Sample, TAD = internal 6 TPB
    AD1CON2 = 0;
    AD1CON1SET = 0x8000; // turn on the ADC
    AD1CHSbits.CH0SA = CH;
    AD1CON1SET = 0x0002; // start sampling ...
    DelayUs(2); // for 2 us
    AD1CON1CLR = 0x0002; // start Converting
    while (!(AD1CON1 & 0x0001)); // conversion done?
    return ADC1BUF0;
}
\end{lstlisting}